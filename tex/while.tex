\chapter{El lenguaje while}
\section{Ejercicio 1}
\begin{enun}
    Siendo la sintaxis para $n$
    \begin{lstlisting}
                    n ::= 0 | 1 | 0 n | 1 n
    \end{lstlisting} 
    ¿Se puede definir $\mathcal{N}$ correctamente?
\end{enun}
\begin{sol}
    La semántica $\mathcal{N}$ para esta sintaxis se podría obtener trivialmente
    utilizando la misma que para el caso en que los constructores compuestos
    tuviesen el orden inverso. De esta forma, lo único que cambiaría sería que
    "la lectura" del número se haría de izquierda a derecha. Si intuitivamente
    pensamos en $n$ como un número binario, diríamos que los bits menos
    significativos son los de la izquierda en contra de lo usual (que son los de
    la derecha).

    Sin embargo, si queremos mantener el convenio de lectura de estos numerales,
    debemos hacer uso de una función auxiliar. Esta se deberá definir
    composicionalmente y simplemente es la longitud:
    \begin{gather*}
        \mathrm{long}: \mathrm{Num} \rightarrow \mathbb{Z}\\
        \mathrm{long}[[0]] = \mathrm{long}[[1]] = 1\\
        \mathrm{long}[[0\ n]] = \mathrm{long}[[1\ n]] = 1 + \mathrm{long}[[n]]
    \end{gather*}
    En definitiva, la función semántica será:
    \begin{gather*}
        \mathcal{N}: \mathrm{Num} \rightarrow \mathbb{Z}\\
        \mathcal{N}[[0]] = 0\\
        \mathcal{N}[[0]] = 1\\
        \mathcal{N}[[0\ n]] = \mathcal{N}[[n]]\\
        \mathcal{N}[[1\ n]] = 2^{\mathrm{long}[[n]]} \cdot \mathcal{N}[[n]]
    \end{gather*}
\end{sol}
