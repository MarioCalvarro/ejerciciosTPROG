\chapter{El lenguaje while}
\section{Ejercicio 1}
\begin{enun}
    Siendo la sintaxis para $n$
    \begin{lstlisting}
                    n ::= 0 | 1 | 0 n | 1 n
    \end{lstlisting} 
    ¿Se puede definir $\mathcal{N}$ correctamente?
\end{enun}
\begin{sol}
    La semántica $\mathcal{N}$ para esta sintaxis se podría obtener trivialmente
    utilizando la misma que para el caso en que los constructores compuestos
    tuviesen el orden inverso. De esta forma, lo único que cambiaría sería que
    "la lectura" del número se haría de izquierda a derecha. Si intuitivamente
    pensamos en $n$ como un número binario, diríamos que los bits menos
    significativos son los de la izquierda en contra de lo usual (que son los de
    la derecha).

    Sin embargo, si queremos mantener el convenio de lectura de estos numerales,
    debemos hacer uso de una función auxiliar. Esta se deberá definir
    composicionalmente y simplemente es la longitud:
    \begin{gather*}
        \mathrm{long}: \mathrm{Num} \rightarrow \mathbb{Z}\\
        \mathrm{long}[[0]] = \mathrm{long}[[1]] = 1\\
        \mathrm{long}[[0\ n]] = \mathrm{long}[[1\ n]] = 1 + \mathrm{long}[[n]]
    \end{gather*}
    En definitiva, la función semántica será:
    \begin{gather*}
        \mathcal{N}: \mathrm{Num} \rightarrow \mathbb{Z}\\
        \mathcal{N}[[0]] = 0\\
        \mathcal{N}[[0]] = 1\\
        \mathcal{N}[[0\ n]] = \mathcal{N}[[n]]\\
        \mathcal{N}[[1\ n]] = 2^{\mathrm{long}[[n]]} \cdot \mathcal{N}[[n]]
    \end{gather*}
\end{sol}

\section{Ejercicio 2}
\begin{enun}
    Explicitar la equivalencia sintáctica entre las dos formas vistas de definir
    sintácticamente los numerales (en un orden y el otro) mediante dos
    biyecciones, $M: \mathrm{Num}' \rightarrow \mathrm{Num}$ y
    $M': \mathrm{Num} \rightarrow \mathrm{Num}'$, tales que $M' \circ M =
    \mathrm{Id}_{\mathrm{Num}'}$ y $M \circ M' = \mathrm{Id}_{\mathrm{Num}}$.
\end{enun}
\begin{sol}
    En primer lugar, de manera natural podemos definir las funciones $M$ y $M'$
    tal que:
    \begin{align*}
        \begin{gathered}
            M: \mathrm{Num}' \rightarrow \mathrm{Num}\\
            M\left( 0' \right) = 0\\
            M\left( 1' \right) = 1\\
            M\left( 0'\ n' \right) = M\left( n' \right)\ 0\\
            M\left( 1'\ n' \right) = M\left( n' \right)\ 1
        \end{gathered}
        &&
        \begin{gathered}
            M': \mathrm{Num} \rightarrow \mathrm{Num}'\\
            M'\left( 0 \right) = 0'\\
            M'\left( 1 \right) = 1'\\
            M'\left( n\ 0 \right) = 0'\ M'\left( n \right)\\
            M'\left( n\ 1 \right) = 1'\ M'\left( n \right)
        \end{gathered}
    \end{align*}
    Veamos ahora por inducción que $M \circ M' = \mathrm{Id}_{\mathrm{Num}}$:
    \begin{enumerate}
        \item \underline{Casos base}:
        \begin{itemize}
            \item $n = 0$: $M'\left( 0 \right) = 0'$ y $M\left( 0' \right) = 0
                \Rightarrow M \circ M' \left( 0 \right) = 0$.
            \item $n = 1$: $M'\left( 1 \right) = 1'$ y $M\left( 1'
                \right) = 1 \Rightarrow M \circ M' \left( 1 \right) = 1$.
        \end{itemize}
        Por tanto, se cumple para los casos base.

        \item \underline{Caso inductivo}: Supongamos por hipótesis de inducción
            que esta composición es la identidad al aplicarse sobre los
            constituyentes de los constructores compuestos de $\mathrm{Num}$,
            que denominaremos $n$. Es decir, $M \circ M' \left( n \right) = n$.
            Veamos ahora que ocurre ahora en los operadores:
            \begin{itemize}
                \item \underline{$n := n\ 0$}: Como $M'\left( n\ 0 \right) = 0'\ M'\left( n
                    \right)$, aplicando ahora $M$ tenemos que $M\left( 0'\
                    M'\left( n \right) \right) = M\left( M'\left( n \right)
                    \right)\ 0 = n\ 0$ (esta última igualdad por hipótesis de
                    inducción).

                \item \underline{$n := n\ 1$}: Similar al anterior punto.
            \end{itemize}
    \end{enumerate}
    Para ver la composición inversa podemos utilizar el mismo razonamiento y con
    esto tenemos el resultado que buscábamos.
\end{sol}

\section{Ejercicio 3}
\begin{enun}
    Definir el conjunto de variables libres para una expresión booleana y
    demostrar un resultado similar al Lema 1.
\end{enun}
\begin{sol}
    Utilizaremos una definición inductiva sobre las expresiones:
    \begin{align*}
        \mathrm{FV}\left( \lstinline{true} \right) &= \mathrm{FV}\left(
        \lstinline{false} \right) = \emptyset\\
        \mathrm{FV}\left( a_1 = a_2 \right) &= \mathrm{FV}\left( a_1 \right) \cup
        \mathrm{FV}\left( a_2 \right)\\
        \mathrm{FV}\left( a_1 \le a_2 \right) &= \mathrm{FV}\left( a_1 \right) \cup
        \mathrm{FV}\left( a_2 \right)\\
        \mathrm{FV}\left( \neg b \right) &= \mathrm{FV}\left( b \right)\\
        \mathrm{FV}\left( b_1 \land b_2 \right) &= \mathrm{FV}\left( b_1 \right) \cup
        \mathrm{FV}\left( b_2 \right)\\
    \end{align*}

    El resultado similar al Lema 1 será entonces:
    \begin{lema}
        Sean $s, s' \in \mathrm{State}$ tales que $\forall x \in
        \mathrm{FV}\left( b \right).s\ x = s'\ x$, entonces $\mathcal{B}\left[
        \left[ b \right] \right]s = \mathcal{B}[[b]]s'$.
    \end{lema}
    \begin{demo}
        Realizaremos una demostración por inducción estructural.
        \begin{itemize}
        \item \underline{Casos base}: ($b := \lstinline{true}$ ó
        $\lstinline{false}$)

        En este caso, $\mathcal{B}[[b]]\text{s} = \text{tt}$ ó $\text{ff}$, respectivamente, $\forall
        s \in \mathrm{State}$.

        \item \underline{Casos inductivos}: Supongamos, por hipótesis de
            inducción, que se cumple el lema para los componentes de los
            constructores compuestos (que serán $a_1, a_2, b, b_1, b_2$),
            entonces:
            \begin{enumerate}
                \item $\boxed{b := a_1 = a_2}$. Sabemos que en este caso,
                    \[
                        \mathcal{B}[[b]]s = \begin{cases}
                            \text{tt},\quad &\text{si } \mathcal{A}[[a_1]]s =
                            \mathcal{A}[[a_2]]s\\
                            \text{ff},\quad &\text{c.c.}
                        \end{cases}
                    \]
                    Por tanto, como $\mathrm{FV}\left( b \right) =
                    \mathrm{FV}\left( a_1 \right) \cup \mathrm{FV}\left( a_2
                    \right)$ y por la hipótesis del lema, tenemos que $a_1$ y $a_2$
                    cumplen la hipótesis del lema 1 y, es decir,
                    $\mathcal{A}[[a_i]]s = \mathcal{A}[[a_i]]s',\ \forall i \in \left\{
                    1, 2\right\}$. Con esto último, podemos simplemente
                    sustituir en la definición de la semántica $s$ por
                    $s'$ y vemos que el resultado es directo.

                \item $\boxed{b := a_1 \le a_2}$. Razonamiento similar al
                    anterior apartado.

                \item $\boxed{b := \neg b_1}$. Recordando la definición de la
                    semántica de este operador tenemos que:
                    \[
                        \mathcal{B}[[\neg b_1]]s = \begin{cases}
                            \text{tt},\quad &\text{si } \mathcal{B}[[b_1]]s =
                            \text{ff}\\
                            \text{ff},\quad &\text{si } \mathcal{B}[[b_1]]s =
                            \text{tt}\\
                        \end{cases}
                    \]
                    Pero como podemos aplicar la hipótesis de inducción sobre
                    $b_1$, tenemos que $\mathcal{B}[[b_1]]s =
                    \mathcal{B}[[b_1]]s'$ y, por tanto, podemos simplemente
                    sustituir en la definición de la semántica $s$ por $s'$ y
                    tendremos el resultado buscado.

                \item $\boxed{b := b_1 \land b_2}$. La definición de la
                    semántica para esta expresión, recordemos, es:
                    \[
                        \mathcal{B}[[b_1 \land b_2]]s = \begin{cases}
                            \text{tt},\quad &\text{si } \mathcal{B}[[b_1]]s =
                            \text{tt y }\mathcal{B}[[b_2]]s = \text{tt}\\
                            \text{ff},\quad &\text{c.c}
                        \end{cases}
                    \]
                    De nuevo, simplemente podemos utilizar la hipótesis de
                    inducción sobre $b_1$ y $b_2$ y, sustituyendo $s$ por $s'$,
                    tendremos el resultado que buscamos.
            \end{enumerate}
        \end{itemize}
    \end{demo}
\end{sol}
